\documentclass[]{article}
\usepackage{lmodern}
\usepackage{amssymb,amsmath}
\usepackage{ifxetex,ifluatex}
\usepackage{fixltx2e} % provides \textsubscript
\ifnum 0\ifxetex 1\fi\ifluatex 1\fi=0 % if pdftex
  \usepackage[T1]{fontenc}
  \usepackage[utf8]{inputenc}
\else % if luatex or xelatex
  \ifxetex
    \usepackage{mathspec}
  \else
    \usepackage{fontspec}
  \fi
  \defaultfontfeatures{Ligatures=TeX,Scale=MatchLowercase}
\fi
% use upquote if available, for straight quotes in verbatim environments
\IfFileExists{upquote.sty}{\usepackage{upquote}}{}
% use microtype if available
\IfFileExists{microtype.sty}{%
\usepackage{microtype}
\UseMicrotypeSet[protrusion]{basicmath} % disable protrusion for tt fonts
}{}
\usepackage[margin=1in]{geometry}
\usepackage{hyperref}
\hypersetup{unicode=true,
            pdftitle={Project 3},
            pdfauthor={Keran Li, Mingming Liu, Zhongxing Xue, Yuhan Zha, Junkai Zhang},
            pdfborder={0 0 0},
            breaklinks=true}
\urlstyle{same}  % don't use monospace font for urls
\usepackage{color}
\usepackage{fancyvrb}
\newcommand{\VerbBar}{|}
\newcommand{\VERB}{\Verb[commandchars=\\\{\}]}
\DefineVerbatimEnvironment{Highlighting}{Verbatim}{commandchars=\\\{\}}
% Add ',fontsize=\small' for more characters per line
\usepackage{framed}
\definecolor{shadecolor}{RGB}{248,248,248}
\newenvironment{Shaded}{\begin{snugshade}}{\end{snugshade}}
\newcommand{\KeywordTok}[1]{\textcolor[rgb]{0.13,0.29,0.53}{\textbf{#1}}}
\newcommand{\DataTypeTok}[1]{\textcolor[rgb]{0.13,0.29,0.53}{#1}}
\newcommand{\DecValTok}[1]{\textcolor[rgb]{0.00,0.00,0.81}{#1}}
\newcommand{\BaseNTok}[1]{\textcolor[rgb]{0.00,0.00,0.81}{#1}}
\newcommand{\FloatTok}[1]{\textcolor[rgb]{0.00,0.00,0.81}{#1}}
\newcommand{\ConstantTok}[1]{\textcolor[rgb]{0.00,0.00,0.00}{#1}}
\newcommand{\CharTok}[1]{\textcolor[rgb]{0.31,0.60,0.02}{#1}}
\newcommand{\SpecialCharTok}[1]{\textcolor[rgb]{0.00,0.00,0.00}{#1}}
\newcommand{\StringTok}[1]{\textcolor[rgb]{0.31,0.60,0.02}{#1}}
\newcommand{\VerbatimStringTok}[1]{\textcolor[rgb]{0.31,0.60,0.02}{#1}}
\newcommand{\SpecialStringTok}[1]{\textcolor[rgb]{0.31,0.60,0.02}{#1}}
\newcommand{\ImportTok}[1]{#1}
\newcommand{\CommentTok}[1]{\textcolor[rgb]{0.56,0.35,0.01}{\textit{#1}}}
\newcommand{\DocumentationTok}[1]{\textcolor[rgb]{0.56,0.35,0.01}{\textbf{\textit{#1}}}}
\newcommand{\AnnotationTok}[1]{\textcolor[rgb]{0.56,0.35,0.01}{\textbf{\textit{#1}}}}
\newcommand{\CommentVarTok}[1]{\textcolor[rgb]{0.56,0.35,0.01}{\textbf{\textit{#1}}}}
\newcommand{\OtherTok}[1]{\textcolor[rgb]{0.56,0.35,0.01}{#1}}
\newcommand{\FunctionTok}[1]{\textcolor[rgb]{0.00,0.00,0.00}{#1}}
\newcommand{\VariableTok}[1]{\textcolor[rgb]{0.00,0.00,0.00}{#1}}
\newcommand{\ControlFlowTok}[1]{\textcolor[rgb]{0.13,0.29,0.53}{\textbf{#1}}}
\newcommand{\OperatorTok}[1]{\textcolor[rgb]{0.81,0.36,0.00}{\textbf{#1}}}
\newcommand{\BuiltInTok}[1]{#1}
\newcommand{\ExtensionTok}[1]{#1}
\newcommand{\PreprocessorTok}[1]{\textcolor[rgb]{0.56,0.35,0.01}{\textit{#1}}}
\newcommand{\AttributeTok}[1]{\textcolor[rgb]{0.77,0.63,0.00}{#1}}
\newcommand{\RegionMarkerTok}[1]{#1}
\newcommand{\InformationTok}[1]{\textcolor[rgb]{0.56,0.35,0.01}{\textbf{\textit{#1}}}}
\newcommand{\WarningTok}[1]{\textcolor[rgb]{0.56,0.35,0.01}{\textbf{\textit{#1}}}}
\newcommand{\AlertTok}[1]{\textcolor[rgb]{0.94,0.16,0.16}{#1}}
\newcommand{\ErrorTok}[1]{\textcolor[rgb]{0.64,0.00,0.00}{\textbf{#1}}}
\newcommand{\NormalTok}[1]{#1}
\usepackage{graphicx,grffile}
\makeatletter
\def\maxwidth{\ifdim\Gin@nat@width>\linewidth\linewidth\else\Gin@nat@width\fi}
\def\maxheight{\ifdim\Gin@nat@height>\textheight\textheight\else\Gin@nat@height\fi}
\makeatother
% Scale images if necessary, so that they will not overflow the page
% margins by default, and it is still possible to overwrite the defaults
% using explicit options in \includegraphics[width, height, ...]{}
\setkeys{Gin}{width=\maxwidth,height=\maxheight,keepaspectratio}
\IfFileExists{parskip.sty}{%
\usepackage{parskip}
}{% else
\setlength{\parindent}{0pt}
\setlength{\parskip}{6pt plus 2pt minus 1pt}
}
\setlength{\emergencystretch}{3em}  % prevent overfull lines
\providecommand{\tightlist}{%
  \setlength{\itemsep}{0pt}\setlength{\parskip}{0pt}}
\setcounter{secnumdepth}{0}
% Redefines (sub)paragraphs to behave more like sections
\ifx\paragraph\undefined\else
\let\oldparagraph\paragraph
\renewcommand{\paragraph}[1]{\oldparagraph{#1}\mbox{}}
\fi
\ifx\subparagraph\undefined\else
\let\oldsubparagraph\subparagraph
\renewcommand{\subparagraph}[1]{\oldsubparagraph{#1}\mbox{}}
\fi

%%% Use protect on footnotes to avoid problems with footnotes in titles
\let\rmarkdownfootnote\footnote%
\def\footnote{\protect\rmarkdownfootnote}

%%% Change title format to be more compact
\usepackage{titling}

% Create subtitle command for use in maketitle
\newcommand{\subtitle}[1]{
  \posttitle{
    \begin{center}\large#1\end{center}
    }
}

\setlength{\droptitle}{-2em}
  \title{Project 3}
  \pretitle{\vspace{\droptitle}\centering\huge}
  \posttitle{\par}
  \author{Keran Li, Mingming Liu, Zhongxing Xue, Yuhan Zha, Junkai Zhang}
  \preauthor{\centering\large\emph}
  \postauthor{\par}
  \predate{\centering\large\emph}
  \postdate{\par}
  \date{March 19, 2018}


\begin{document}
\maketitle

\subsubsection{Step 0: Prepare packages}\label{step-0-prepare-packages}

\begin{Shaded}
\begin{Highlighting}[]
\CommentTok{# Load the libraries}
\NormalTok{packages.used <-}\StringTok{ }\KeywordTok{c}\NormalTok{(}\StringTok{"EBImage"}\NormalTok{,}\StringTok{"gbm"}\NormalTok{,}\StringTok{"caret"}\NormalTok{,}\StringTok{"nnet"}\NormalTok{,}\StringTok{"xgboost"}\NormalTok{,}\StringTok{"xgboost"}\NormalTok{,}
                   \StringTok{"haven"}\NormalTok{,}\StringTok{"tidyverse"}\NormalTok{,}\StringTok{"mlr"}\NormalTok{,}\StringTok{"plyr"}\NormalTok{,}\StringTok{"ggplot2"}\NormalTok{,}\StringTok{"randomForest"}\NormalTok{,}
                   \StringTok{"adabag"}\NormalTok{,}\StringTok{"neuralnet"}\NormalTok{,}\StringTok{"grDevices"}\NormalTok{,}\StringTok{"kernlab"}\NormalTok{,}\StringTok{"tree"}\NormalTok{)}

\CommentTok{#if(!require("EBImage"))\{}
\CommentTok{#  source("https://bioconductor.org/biocLite.R")}
\CommentTok{#  biocLite("EBImage")}
\CommentTok{#\}}

\CommentTok{# Check packages}
\NormalTok{packages.needed=}\KeywordTok{setdiff}\NormalTok{(packages.used, }
                        \KeywordTok{intersect}\NormalTok{(}\KeywordTok{installed.packages}\NormalTok{()[,}\DecValTok{1}\NormalTok{], }
\NormalTok{                                  packages.used))}
\CommentTok{# install needed packages}
\ControlFlowTok{if}\NormalTok{(}\KeywordTok{length}\NormalTok{(packages.needed)}\OperatorTok{>}\DecValTok{0}\NormalTok{)\{}
  \KeywordTok{install.packages}\NormalTok{(packages.needed, }\DataTypeTok{dependencies =} \OtherTok{TRUE}\NormalTok{)}
\NormalTok{\}}

\KeywordTok{library}\NormalTok{(}\StringTok{"EBImage"}\NormalTok{)}
\end{Highlighting}
\end{Shaded}

\begin{verbatim}
## Warning: package 'EBImage' was built under R version 3.4.3
\end{verbatim}

\begin{Shaded}
\begin{Highlighting}[]
\KeywordTok{library}\NormalTok{(}\StringTok{"gbm"}\NormalTok{)}
\end{Highlighting}
\end{Shaded}

\begin{verbatim}
## Loading required package: survival
\end{verbatim}

\begin{verbatim}
## Loading required package: lattice
\end{verbatim}

\begin{verbatim}
## Loading required package: splines
\end{verbatim}

\begin{verbatim}
## Loading required package: parallel
\end{verbatim}

\begin{verbatim}
## Loaded gbm 2.1.3
\end{verbatim}

\begin{Shaded}
\begin{Highlighting}[]
\KeywordTok{library}\NormalTok{(}\StringTok{"kernlab"}\NormalTok{)}
\KeywordTok{library}\NormalTok{(}\StringTok{"tree"}\NormalTok{)}
\end{Highlighting}
\end{Shaded}

\begin{verbatim}
## Warning: package 'tree' was built under R version 3.4.4
\end{verbatim}

\begin{Shaded}
\begin{Highlighting}[]
\KeywordTok{library}\NormalTok{(}\StringTok{"caret"}\NormalTok{)}
\end{Highlighting}
\end{Shaded}

\begin{verbatim}
## Warning: package 'caret' was built under R version 3.4.3
\end{verbatim}

\begin{verbatim}
## Loading required package: ggplot2
\end{verbatim}

\begin{verbatim}
## 
## Attaching package: 'ggplot2'
\end{verbatim}

\begin{verbatim}
## The following object is masked from 'package:kernlab':
## 
##     alpha
\end{verbatim}

\begin{verbatim}
## 
## Attaching package: 'caret'
\end{verbatim}

\begin{verbatim}
## The following object is masked from 'package:survival':
## 
##     cluster
\end{verbatim}

\begin{Shaded}
\begin{Highlighting}[]
\KeywordTok{library}\NormalTok{(}\StringTok{"nnet"}\NormalTok{)}
\KeywordTok{library}\NormalTok{(}\StringTok{"xgboost"}\NormalTok{)}
\end{Highlighting}
\end{Shaded}

\begin{verbatim}
## Warning: package 'xgboost' was built under R version 3.4.3
\end{verbatim}

\begin{Shaded}
\begin{Highlighting}[]
\KeywordTok{library}\NormalTok{(}\StringTok{"haven"}\NormalTok{)}
\end{Highlighting}
\end{Shaded}

\begin{verbatim}
## Warning: package 'haven' was built under R version 3.4.3
\end{verbatim}

\begin{Shaded}
\begin{Highlighting}[]
\KeywordTok{library}\NormalTok{(}\StringTok{"tidyverse"}\NormalTok{) }
\end{Highlighting}
\end{Shaded}

\begin{verbatim}
## -- Attaching packages ----------------------------------------------------- tidyverse 1.2.1 --
\end{verbatim}

\begin{verbatim}
## √ tibble  1.4.2     √ purrr   0.2.4
## √ tidyr   0.8.0     √ dplyr   0.7.4
## √ readr   1.1.1     √ stringr 1.3.0
## √ tibble  1.4.2     √ forcats 0.3.0
\end{verbatim}

\begin{verbatim}
## Warning: package 'tibble' was built under R version 3.4.3
\end{verbatim}

\begin{verbatim}
## Warning: package 'tidyr' was built under R version 3.4.3
\end{verbatim}

\begin{verbatim}
## Warning: package 'stringr' was built under R version 3.4.3
\end{verbatim}

\begin{verbatim}
## Warning: package 'forcats' was built under R version 3.4.3
\end{verbatim}

\begin{verbatim}
## -- Conflicts -------------------------------------------------------- tidyverse_conflicts() --
## x ggplot2::alpha()   masks kernlab::alpha()
## x dplyr::combine()   masks EBImage::combine()
## x purrr::cross()     masks kernlab::cross()
## x dplyr::filter()    masks stats::filter()
## x dplyr::lag()       masks stats::lag()
## x purrr::lift()      masks caret::lift()
## x dplyr::slice()     masks xgboost::slice()
## x purrr::transpose() masks EBImage::transpose()
\end{verbatim}

\begin{Shaded}
\begin{Highlighting}[]
\KeywordTok{library}\NormalTok{(}\StringTok{"mlr"}\NormalTok{)}
\end{Highlighting}
\end{Shaded}

\begin{verbatim}
## Loading required package: ParamHelpers
\end{verbatim}

\begin{verbatim}
## 
## Attaching package: 'mlr'
\end{verbatim}

\begin{verbatim}
## The following object is masked from 'package:caret':
## 
##     train
\end{verbatim}

\begin{Shaded}
\begin{Highlighting}[]
\KeywordTok{library}\NormalTok{(}\StringTok{"plyr"}\NormalTok{)}
\end{Highlighting}
\end{Shaded}

\begin{verbatim}
## -------------------------------------------------------------------------
\end{verbatim}

\begin{verbatim}
## You have loaded plyr after dplyr - this is likely to cause problems.
## If you need functions from both plyr and dplyr, please load plyr first, then dplyr:
## library(plyr); library(dplyr)
\end{verbatim}

\begin{verbatim}
## -------------------------------------------------------------------------
\end{verbatim}

\begin{verbatim}
## 
## Attaching package: 'plyr'
\end{verbatim}

\begin{verbatim}
## The following objects are masked from 'package:dplyr':
## 
##     arrange, count, desc, failwith, id, mutate, rename, summarise,
##     summarize
\end{verbatim}

\begin{verbatim}
## The following object is masked from 'package:purrr':
## 
##     compact
\end{verbatim}

\begin{Shaded}
\begin{Highlighting}[]
\KeywordTok{library}\NormalTok{(}\StringTok{"ggplot2"}\NormalTok{)}
\KeywordTok{library}\NormalTok{(}\StringTok{"randomForest"}\NormalTok{)}
\end{Highlighting}
\end{Shaded}

\begin{verbatim}
## Warning: package 'randomForest' was built under R version 3.4.4
\end{verbatim}

\begin{verbatim}
## randomForest 4.6-14
\end{verbatim}

\begin{verbatim}
## Type rfNews() to see new features/changes/bug fixes.
\end{verbatim}

\begin{verbatim}
## 
## Attaching package: 'randomForest'
\end{verbatim}

\begin{verbatim}
## The following object is masked from 'package:dplyr':
## 
##     combine
\end{verbatim}

\begin{verbatim}
## The following object is masked from 'package:ggplot2':
## 
##     margin
\end{verbatim}

\begin{verbatim}
## The following object is masked from 'package:EBImage':
## 
##     combine
\end{verbatim}

\begin{Shaded}
\begin{Highlighting}[]
\KeywordTok{library}\NormalTok{(}\StringTok{"adabag"}\NormalTok{)}
\end{Highlighting}
\end{Shaded}

\begin{verbatim}
## Warning: package 'adabag' was built under R version 3.4.3
\end{verbatim}

\begin{verbatim}
## Loading required package: rpart
\end{verbatim}

\begin{verbatim}
## Warning: package 'rpart' was built under R version 3.4.3
\end{verbatim}

\begin{verbatim}
## Loading required package: foreach
\end{verbatim}

\begin{verbatim}
## Warning: package 'foreach' was built under R version 3.4.3
\end{verbatim}

\begin{verbatim}
## 
## Attaching package: 'foreach'
\end{verbatim}

\begin{verbatim}
## The following objects are masked from 'package:purrr':
## 
##     accumulate, when
\end{verbatim}

\begin{verbatim}
## Loading required package: doParallel
\end{verbatim}

\begin{verbatim}
## Loading required package: iterators
\end{verbatim}

\begin{verbatim}
## Warning: package 'iterators' was built under R version 3.4.3
\end{verbatim}

\begin{Shaded}
\begin{Highlighting}[]
\KeywordTok{library}\NormalTok{(}\StringTok{"neuralnet"}\NormalTok{)}
\end{Highlighting}
\end{Shaded}

\begin{verbatim}
## 
## Attaching package: 'neuralnet'
\end{verbatim}

\begin{verbatim}
## The following object is masked from 'package:dplyr':
## 
##     compute
\end{verbatim}

\begin{Shaded}
\begin{Highlighting}[]
\KeywordTok{library}\NormalTok{(}\StringTok{"grDevices"}\NormalTok{)}
\end{Highlighting}
\end{Shaded}

\subsubsection{Step 0: specify
directories.}\label{step-0-specify-directories.}

Set the working directory to the image folder. Specify the training and
the testing set. For data without an independent test/validation set,
you need to create your own testing data by random subsampling. In order
to obain reproducible results, set.seed() whenever randomization is
used.

\begin{Shaded}
\begin{Highlighting}[]
\CommentTok{#setwd("~/Desktop/Spring2018-Project3-Group3/")}
\end{Highlighting}
\end{Shaded}

\begin{Shaded}
\begin{Highlighting}[]
\NormalTok{experiment_dir <-}\StringTok{ "../data/"}
\NormalTok{img_train_dir <-}\StringTok{ }\KeywordTok{paste}\NormalTok{(experiment_dir, }\StringTok{"train/"}\NormalTok{, }\DataTypeTok{sep=}\StringTok{""}\NormalTok{)}
\NormalTok{img_test_dir <-}\StringTok{ }\KeywordTok{paste}\NormalTok{(experiment_dir, }\StringTok{"test/"}\NormalTok{, }\DataTypeTok{sep=}\StringTok{""}\NormalTok{)}
\end{Highlighting}
\end{Shaded}

\subsubsection{Step 1: Set up controls for evaluation
experiments.}\label{step-1-set-up-controls-for-evaluation-experiments.}

In this chunk, we have a set of controls for the evaluation experiments.

(T/F) process features for training set (T/F) run evaluation on an
independent test set (T/F) process features for test set

\begin{Shaded}
\begin{Highlighting}[]
\NormalTok{run.feature.train =}\StringTok{ }\OtherTok{TRUE}    
\NormalTok{run.test =}\StringTok{ }\OtherTok{TRUE}
\NormalTok{run.feature.test =}\StringTok{ }\OtherTok{TRUE} 
\NormalTok{run.feature.RGB <-}\StringTok{ }\OtherTok{FALSE}
\end{Highlighting}
\end{Shaded}

\subparagraph{Set up controls indicating which model to
run}\label{set-up-controls-indicating-which-model-to-run}

\begin{Shaded}
\begin{Highlighting}[]
\NormalTok{run.xgboost =}\StringTok{ }\OtherTok{FALSE}
\NormalTok{run.adaboost =}\StringTok{ }\OtherTok{FALSE}
\NormalTok{run.tree =}\StringTok{ }\OtherTok{FALSE}
\NormalTok{run.gbm =}\StringTok{ }\OtherTok{TRUE}
\NormalTok{run.lg =}\StringTok{ }\OtherTok{FALSE}
\NormalTok{run.nnt =}\StringTok{ }\OtherTok{FALSE}
\NormalTok{run.rf =}\StringTok{ }\OtherTok{FALSE}
\NormalTok{run.svm =}\StringTok{ }\OtherTok{FALSE}
\end{Highlighting}
\end{Shaded}

\subsubsection{Step 2: Extract features}\label{step-2-extract-features}

\begin{Shaded}
\begin{Highlighting}[]
\KeywordTok{source}\NormalTok{(}\StringTok{"../lib/SelectFeature.R"}\NormalTok{)}

\CommentTok{# tm_feature_train <- NA}
\CommentTok{# if(run.feature.train)\{}
\CommentTok{#   tm_feature_train <- system.time(dat_train <- SelectFeature(SIFTname, SubGroup))}
\CommentTok{# \}}
\CommentTok{# cat("Time for constructing training features=", tm_feature_train[1], "s \textbackslash{}n")}

\CommentTok{#tm_feature_test <- NA}
\CommentTok{#if(run.feature.test)\{}
\CommentTok{#  tm_feature_test <- system.time(dat_test <-SelectFeature())}
\CommentTok{#\}}
\CommentTok{# Running time}
\CommentTok{#cat("Time for constructing testing features=", tm_feature_test[1], "s \textbackslash{}n")}

\CommentTok{#save(dat_train, file="./output/feature_train.RData")}
\CommentTok{#save(dat_test, file="./output/feature_test.RData")}
\ControlFlowTok{if}\NormalTok{ (run.feature.RGB)}
\NormalTok{\{}
\NormalTok{  TotalImage <-}\StringTok{ }\DecValTok{2100}
\NormalTok{  lottery <-}\StringTok{ }\KeywordTok{sample}\NormalTok{(}\DecValTok{1}\OperatorTok{:}\DecValTok{5}\NormalTok{, TotalImage, }\DataTypeTok{replace =} \OtherTok{TRUE}\NormalTok{)}
\NormalTok{  FeatureData <-}\StringTok{ }\KeywordTok{SelectFeature}\NormalTok{(}\DataTypeTok{N =}\NormalTok{ TotalImage)}
\NormalTok{  class_train <-}\StringTok{ }\KeywordTok{c}\NormalTok{(}\KeywordTok{rep}\NormalTok{(}\DecValTok{1}\NormalTok{,}\DecValTok{1000}\NormalTok{), }\KeywordTok{rep}\NormalTok{(}\DecValTok{2}\NormalTok{, }\DecValTok{1000}\NormalTok{), }\KeywordTok{rep}\NormalTok{(}\DecValTok{3}\NormalTok{, }\DecValTok{1000}\NormalTok{))}
\NormalTok{  class_train <-}\StringTok{ }\NormalTok{class_train[}\DecValTok{1}\OperatorTok{:}\NormalTok{TotalImage]}
\NormalTok{  dat_train <-}\StringTok{ }\KeywordTok{cbind}\NormalTok{(class_train[}\KeywordTok{which}\NormalTok{(lottery }\OperatorTok{!=}\StringTok{ }\DecValTok{1}\NormalTok{)], FeatureData[}\KeywordTok{which}\NormalTok{(lottery }\OperatorTok{!=}\StringTok{ }\DecValTok{1}\NormalTok{),])}
  \KeywordTok{colnames}\NormalTok{(dat_train)[}\DecValTok{1}\NormalTok{] <-}\StringTok{ "class"}
\NormalTok{  dat_test <-}\StringTok{  }\NormalTok{FeatureData[}\KeywordTok{which}\NormalTok{(lottery }\OperatorTok{==}\StringTok{ }\DecValTok{1}\NormalTok{),]}
\NormalTok{\}}
\ControlFlowTok{if}\NormalTok{ (}\OperatorTok{!}\NormalTok{run.feature.RGB)}
\NormalTok{\{}
\NormalTok{  TotalImage <-}\StringTok{ }\DecValTok{3000}
\NormalTok{  lottery <-}\StringTok{ }\KeywordTok{sample}\NormalTok{(}\DecValTok{1}\OperatorTok{:}\DecValTok{5}\NormalTok{, TotalImage, }\DataTypeTok{replace =} \OtherTok{TRUE}\NormalTok{)}
\NormalTok{  class_train <-}\StringTok{ }\KeywordTok{c}\NormalTok{(}\KeywordTok{rep}\NormalTok{(}\DecValTok{1}\NormalTok{,}\DecValTok{1000}\NormalTok{), }\KeywordTok{rep}\NormalTok{(}\DecValTok{2}\NormalTok{, }\DecValTok{1000}\NormalTok{), }\KeywordTok{rep}\NormalTok{(}\DecValTok{3}\NormalTok{, }\DecValTok{1000}\NormalTok{))}
\NormalTok{  dat_train <-}\StringTok{ }\KeywordTok{read.csv}\NormalTok{(}\StringTok{"../lib/SIFT_test.csv"}\NormalTok{)}
\NormalTok{  dat_train <-}\StringTok{ }\NormalTok{dat_train[,}\OperatorTok{-}\DecValTok{1}\NormalTok{]}
\NormalTok{  dat_test <-}\StringTok{ }\NormalTok{dat_train[}\KeywordTok{which}\NormalTok{(lottery }\OperatorTok{==}\StringTok{ }\DecValTok{1}\NormalTok{),]}
\NormalTok{  dat_train <-}\StringTok{ }\KeywordTok{cbind}\NormalTok{(class_train[}\KeywordTok{which}\NormalTok{(lottery }\OperatorTok{!=}\StringTok{ }\DecValTok{1}\NormalTok{)], dat_train[}\KeywordTok{which}\NormalTok{(lottery }\OperatorTok{!=}\StringTok{ }\DecValTok{1}\NormalTok{),])}
  \KeywordTok{colnames}\NormalTok{(dat_train)[}\DecValTok{1}\NormalTok{] <-}\StringTok{ "class"}
\NormalTok{\}}
\end{Highlighting}
\end{Shaded}

\subsubsection{Step 3: Import training and test data of best
features.}\label{step-3-import-training-and-test-data-of-best-features.}

\begin{Shaded}
\begin{Highlighting}[]
\CommentTok{#train_all_data <- rbind(tr_data, te_data)}
\CommentTok{#train_all_class <- rbind(tr_class, te_class)}

\CommentTok{# Training data}
\CommentTok{#dat_train <- cbind(train_all_class[,-1], train_all_data[,-1])}


\CommentTok{# testdata <- cbind(te_class[,-1], te_data[,-1])}
\CommentTok{# colnames(test)[1] <- "class"}
\CommentTok{# dim(dat_train)}
\end{Highlighting}
\end{Shaded}

\subsubsection{Step 4: Train a classification model with training
images}\label{step-4-train-a-classification-model-with-training-images}

Call the train model and test model from library.

\texttt{train.R} and \texttt{test.R} should be wrappers for all your
model training steps and your classification/prediction steps. +
\texttt{train.R} + Input: a path that points to the training set
features. + Input: an R object of training sample labels. + Output: an
RData file that contains trained classifiers in the forms of R objects:
models/settings/links to external trained configurations. +
\texttt{test.R} + Input: a path that points to the test set features. +
Input: an R object that contains a trained classifier. + Output: an R
object of class label predictions on the test set. If there are multiple
classifiers under evaluation, there should be multiple sets of label
predictions.

\begin{Shaded}
\begin{Highlighting}[]
\KeywordTok{source}\NormalTok{(}\StringTok{"../lib/train.R"}\NormalTok{)}
\KeywordTok{source}\NormalTok{(}\StringTok{"../lib/test.R"}\NormalTok{)}
\end{Highlighting}
\end{Shaded}

\begin{itemize}
\tightlist
\item
  Train the model with the entire training set using the selected model.
\end{itemize}

\begin{Shaded}
\begin{Highlighting}[]
\ControlFlowTok{if}\NormalTok{(run.xgboost)\{}
\NormalTok{  dat_train_xgb <-}\StringTok{ }\KeywordTok{as.matrix}\NormalTok{(dat_train)}
\NormalTok{  xgboost_train <-}\StringTok{ }\KeywordTok{train_xgboost}\NormalTok{(dat_train_xgb)}
\NormalTok{  xgbPred <-}\StringTok{ }\KeywordTok{xgboost_test}\NormalTok{(xgboost_train}\OperatorTok{$}\NormalTok{fit, }\KeywordTok{as.matrix}\NormalTok{(dat_test))}
  \KeywordTok{cat}\NormalTok{(}\StringTok{"Accuracy"}\NormalTok{, }\KeywordTok{mean}\NormalTok{(xgbPred }\OperatorTok{==}\StringTok{ }\NormalTok{class_train[}\KeywordTok{which}\NormalTok{(lottery }\OperatorTok{==}\StringTok{ }\DecValTok{1}\NormalTok{)]))}
  \KeywordTok{cat}\NormalTok{(}\StringTok{"Time for testing model="}\NormalTok{, xgboost_train}\OperatorTok{$}\NormalTok{time, }\StringTok{"s }\CharTok{\textbackslash{}n}\StringTok{"}\NormalTok{)}
\NormalTok{\}}
\ControlFlowTok{if}\NormalTok{(run.adaboost)\{}
\NormalTok{  adaboost_train <-}\StringTok{ }\KeywordTok{train_adaboosting}\NormalTok{(dat_train)}
  \KeywordTok{cat}\NormalTok{(}\StringTok{"Time for testing model="}\NormalTok{, adaboost_train}\OperatorTok{$}\NormalTok{time, }\StringTok{"s }\CharTok{\textbackslash{}n}\StringTok{"}\NormalTok{)}
\NormalTok{\}}
\ControlFlowTok{if}\NormalTok{(run.tree)\{}
\NormalTok{  tree_train <-}\StringTok{ }\KeywordTok{train_tree}\NormalTok{(dat_train)}
  \KeywordTok{cat}\NormalTok{(}\StringTok{"Time for testing model="}\NormalTok{, tree_train}\OperatorTok{$}\NormalTok{time, }\StringTok{"s }\CharTok{\textbackslash{}n}\StringTok{"}\NormalTok{)}
\NormalTok{\}}
\ControlFlowTok{if}\NormalTok{(run.gbm)\{}
\NormalTok{  gbm_train <-}\StringTok{  }\KeywordTok{train_GBM}\NormalTok{(dat_train)}
\NormalTok{  gbmPred <-}\StringTok{ }\KeywordTok{gbm_test}\NormalTok{(gbm_train}\OperatorTok{$}\NormalTok{fit, dat_test)}
  \KeywordTok{cat}\NormalTok{(}\StringTok{"Accuracy"}\NormalTok{, }\KeywordTok{mean}\NormalTok{(gbmPred }\OperatorTok{==}\StringTok{ }\NormalTok{class_train[}\KeywordTok{which}\NormalTok{(lottery }\OperatorTok{==}\StringTok{ }\DecValTok{1}\NormalTok{)]))}
  \KeywordTok{cat}\NormalTok{(}\StringTok{"Time for testing model="}\NormalTok{, gbm_train}\OperatorTok{$}\NormalTok{time, }\StringTok{"s }\CharTok{\textbackslash{}n}\StringTok{"}\NormalTok{)}
\NormalTok{\}}
\end{Highlighting}
\end{Shaded}

\begin{verbatim}
## Using test method...
## Accuracy 0.5260586Time for testing model= 1.272167 s
\end{verbatim}

\begin{Shaded}
\begin{Highlighting}[]
\ControlFlowTok{if}\NormalTok{(run.lg)\{}
\NormalTok{  lg_train <-}\StringTok{ }\KeywordTok{train_logistic}\NormalTok{(dat_train)}
  \KeywordTok{cat}\NormalTok{(}\StringTok{"Time for testing model="}\NormalTok{, lg_train}\OperatorTok{$}\NormalTok{time, }\StringTok{"s }\CharTok{\textbackslash{}n}\StringTok{"}\NormalTok{)}
\NormalTok{\}}
\ControlFlowTok{if}\NormalTok{(run.nnt)\{}
\NormalTok{  nnt_train <-}\StringTok{ }\KeywordTok{train_nn}\NormalTok{(dat_train)}
  \KeywordTok{cat}\NormalTok{(}\StringTok{"Time for testing model="}\NormalTok{, nnt_train}\OperatorTok{$}\NormalTok{time, }\StringTok{"s }\CharTok{\textbackslash{}n}\StringTok{"}\NormalTok{)}
\NormalTok{\}}
\ControlFlowTok{if}\NormalTok{(run.rf)\{}
\NormalTok{  rf_train <-}\StringTok{ }\KeywordTok{train_rf}\NormalTok{(dat_train)}
  \KeywordTok{cat}\NormalTok{(}\StringTok{"Time for testing model="}\NormalTok{, rf_train}\OperatorTok{$}\NormalTok{time, }\StringTok{"s }\CharTok{\textbackslash{}n}\StringTok{"}\NormalTok{)}
\NormalTok{\}}
\ControlFlowTok{if}\NormalTok{(run.svm)\{}
\NormalTok{  svm_train <-}\StringTok{ }\KeywordTok{train_svm}\NormalTok{(dat_train)}
  \KeywordTok{cat}\NormalTok{(}\StringTok{"Time for testing model="}\NormalTok{, svm_train}\OperatorTok{$}\NormalTok{time, }\StringTok{"s }\CharTok{\textbackslash{}n}\StringTok{"}\NormalTok{)}
\NormalTok{\}}
\end{Highlighting}
\end{Shaded}

\subsubsection{Step 5: Make prediction and summarize Running
Time}\label{step-5-make-prediction-and-summarize-running-time}

Feed the final training model with the completely holdout testing data.
Prediction performance matters, so does the running times for
constructing features and for training the model, especially when the
computation resource is limited.

\begin{Shaded}
\begin{Highlighting}[]
\NormalTok{tm_xgboost_test=}\OtherTok{NA}
\NormalTok{tm_adaboost_test=}\OtherTok{NA}
\NormalTok{tm_tree_test=}\OtherTok{NA}
\NormalTok{tm_gbm_test=}\OtherTok{NA}
\NormalTok{tm_lg_test=}\OtherTok{NA}
\NormalTok{tm_nnt_test=}\OtherTok{NA}
\NormalTok{tm_rf_test=}\OtherTok{NA}
\NormalTok{tm_svm_test=}\OtherTok{NA}

\ControlFlowTok{if}\NormalTok{(run.xgboost)\{}
  \KeywordTok{load}\NormalTok{(}\DataTypeTok{file=}\StringTok{"../lib/XgBoost/XgBoost.RData"}\NormalTok{)}
\NormalTok{  tm_xgboost_test <-}\StringTok{ }\KeywordTok{system.time}\NormalTok{(xgboost.pred <-}\StringTok{ }\KeywordTok{predict}\NormalTok{(model, }\KeywordTok{as.matrix}\NormalTok{(dat_test)) }\OperatorTok{+}\StringTok{ }\DecValTok{1}\NormalTok{)}
  \KeywordTok{save}\NormalTok{(xgboost.pred, }\DataTypeTok{file=}\StringTok{"../output/xgboost.pred.RData"}\NormalTok{)}
  \KeywordTok{cat}\NormalTok{(}\StringTok{"Accuracy"}\NormalTok{, }\KeywordTok{mean}\NormalTok{(xgboost.pred }\OperatorTok{==}\StringTok{ }\NormalTok{class_train[}\KeywordTok{which}\NormalTok{(lottery }\OperatorTok{==}\StringTok{ }\DecValTok{1}\NormalTok{)]))}
  \KeywordTok{cat}\NormalTok{(}\StringTok{"Time for testing model="}\NormalTok{, tm_xgboost_test[}\DecValTok{1}\NormalTok{], }\StringTok{"s }\CharTok{\textbackslash{}n}\StringTok{"}\NormalTok{)}
\NormalTok{\}}
\ControlFlowTok{if}\NormalTok{(run.adaboost)\{}
  \KeywordTok{load}\NormalTok{(}\DataTypeTok{file=}\StringTok{"../lib/AdaBoost/adaboost.RData"}\NormalTok{)}
\NormalTok{  tm_adaboost_test <-}\StringTok{ }\KeywordTok{system.time}\NormalTok{(ada.pred <-}\StringTok{ }\KeywordTok{predict}\NormalTok{(adaall, dat_test))}
  \KeywordTok{save}\NormalTok{(ada.pred , }\DataTypeTok{file=}\StringTok{"../output/ada.pred .RData"}\NormalTok{)}
  \KeywordTok{cat}\NormalTok{(}\StringTok{"Time for testing model="}\NormalTok{, tm_adaboost_test[}\DecValTok{1}\NormalTok{], }\StringTok{"s }\CharTok{\textbackslash{}n}\StringTok{"}\NormalTok{)}
\NormalTok{\}}
\ControlFlowTok{if}\NormalTok{(run.tree)\{}
  \KeywordTok{load}\NormalTok{(}\DataTypeTok{file=}\StringTok{"../lib/ClassficationTree/tree.RData"}\NormalTok{)}
\NormalTok{  tm_tree_test <-}\StringTok{ }\KeywordTok{system.time}\NormalTok{(tree.pred <-}\StringTok{ }\KeywordTok{predict}\NormalTok{(tree, dat_test))}
  \KeywordTok{save}\NormalTok{(tree.pred, }\DataTypeTok{file=}\StringTok{"../output/tree.pred.RData"}\NormalTok{)}
  \KeywordTok{cat}\NormalTok{(}\StringTok{"Time for testing model="}\NormalTok{, tm_tree_test[}\DecValTok{1}\NormalTok{], }\StringTok{"s }\CharTok{\textbackslash{}n}\StringTok{"}\NormalTok{)}
\NormalTok{\}}
\ControlFlowTok{if}\NormalTok{(run.gbm)\{}
  \KeywordTok{load}\NormalTok{(}\DataTypeTok{file=}\StringTok{"../lib/GBM/GBM.RData"}\NormalTok{)}
\NormalTok{  tm_gbm_test <-}\StringTok{ }\KeywordTok{system.time}\NormalTok{(gbm.pred <-}\StringTok{ }\KeywordTok{predict}\NormalTok{(gbm_train}\OperatorTok{$}\NormalTok{fit, dat_test))}
  \KeywordTok{save}\NormalTok{(gbm.pred, }\DataTypeTok{file=}\StringTok{"../output/gbm.pred.RData"}\NormalTok{)}
  \CommentTok{#cat("Accuracy", mean(gbm.pred == class_train[which(lottery == 1)]))}
  \KeywordTok{cat}\NormalTok{(}\StringTok{"Time for testing model="}\NormalTok{, tm_gbm_test[}\DecValTok{1}\NormalTok{], }\StringTok{"s }\CharTok{\textbackslash{}n}\StringTok{"}\NormalTok{)}
\NormalTok{\}}
\end{Highlighting}
\end{Shaded}

\begin{verbatim}
## Using 19 trees...
## Time for testing model= 0.104 s
\end{verbatim}

\begin{Shaded}
\begin{Highlighting}[]
\ControlFlowTok{if}\NormalTok{(run.lg)\{}
  \KeywordTok{load}\NormalTok{(}\DataTypeTok{file=}\StringTok{"../lib/Logistic/logistic.RData"}\NormalTok{)}
\NormalTok{  tm_lg_test <-}\StringTok{ }\KeywordTok{system.time}\NormalTok{(lg.pred <-}\StringTok{ }\KeywordTok{predict}\NormalTok{(fit, dat_test))}
  \KeywordTok{save}\NormalTok{(lg.pred, }\DataTypeTok{file=}\StringTok{"../output/lg.pred.RData"}\NormalTok{)}
  \KeywordTok{cat}\NormalTok{(}\StringTok{"Time for testing model="}\NormalTok{, tm_lg_test[}\DecValTok{1}\NormalTok{], }\StringTok{"s }\CharTok{\textbackslash{}n}\StringTok{"}\NormalTok{)}
\NormalTok{\}}
\ControlFlowTok{if}\NormalTok{(run.nnt)\{}
  \KeywordTok{load}\NormalTok{(}\DataTypeTok{file=}\StringTok{"../lib/NeuronNetwork/NeuronNetwork.RData"}\NormalTok{)}
\NormalTok{  tm_nnt_test <-}\StringTok{ }\KeywordTok{system.time}\NormalTok{(nnt.pred <-}\StringTok{ }\KeywordTok{predict}\NormalTok{(pr.nn, dat_test))}
  \KeywordTok{save}\NormalTok{(nnt.pred, }\DataTypeTok{file=}\StringTok{"../output/nnt.pred.RData"}\NormalTok{)}
  \KeywordTok{cat}\NormalTok{(}\StringTok{"Time for testing model="}\NormalTok{, tm_nnt_test[}\DecValTok{1}\NormalTok{], }\StringTok{"s }\CharTok{\textbackslash{}n}\StringTok{"}\NormalTok{)}
\NormalTok{\}}
\ControlFlowTok{if}\NormalTok{(run.rf)\{}
  \KeywordTok{load}\NormalTok{(}\DataTypeTok{file=}\StringTok{"../lib/RandomForest/randomforest.RData"}\NormalTok{)}
\NormalTok{  tm_rf_test <-}\StringTok{ }\KeywordTok{system.time}\NormalTok{(rf.pred <-}\StringTok{ }\KeywordTok{predict}\NormalTok{(rf_all, dat_test, }\DataTypeTok{type=}\StringTok{"response"}\NormalTok{))}
  \KeywordTok{save}\NormalTok{(rf.pred , }\DataTypeTok{file=}\StringTok{"../output/rf.pred .RData"}\NormalTok{)}
  \KeywordTok{cat}\NormalTok{(}\StringTok{"Time for testing model="}\NormalTok{, tm_rf_test[}\DecValTok{1}\NormalTok{], }\StringTok{"s }\CharTok{\textbackslash{}n}\StringTok{"}\NormalTok{)}
\NormalTok{\}}
\ControlFlowTok{if}\NormalTok{(run.svm)\{}
  \KeywordTok{load}\NormalTok{(}\DataTypeTok{file=}\StringTok{"../lib/SVM/svm.RData"}\NormalTok{)}
\NormalTok{  tm_svm_test <-}\StringTok{ }\KeywordTok{system.time}\NormalTok{(svm.pred <-}\StringTok{ }\KeywordTok{predict}\NormalTok{(rad_svm_fit, dat_test))}
  \KeywordTok{save}\NormalTok{(svm.pred , }\DataTypeTok{file=}\StringTok{"../output/svm.pred.RData"}\NormalTok{)}
  \KeywordTok{cat}\NormalTok{(}\StringTok{"Time for testing model="}\NormalTok{, tm_svm_test[}\DecValTok{1}\NormalTok{], }\StringTok{"s }\CharTok{\textbackslash{}n}\StringTok{"}\NormalTok{)}
\NormalTok{\}}
\end{Highlighting}
\end{Shaded}


\end{document}
